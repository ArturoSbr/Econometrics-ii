% Imports
\documentclass[12pt]{article}
\usepackage[margin=1in]{geometry} 
\usepackage{
    amsmath, amsthm, amssymb, verbatim
}

% Begin document
\setlength{\parindent}{0pt}
\begin{document}

% Title
\title{Econometrics II Workshop}
\author{Arturo Soberon}
\date{}
\maketitle

% Main

% Bash
\setcounter{section}{0}
\section{Bash}
Bash is a Unix shell used in Linux and Mac OS that lets you interact with
the operating system through typed commands. We will mainly use it to run
git commands, but there are many other useful commands worth noting.

\begin{enumerate}
    \item \texttt{pwd} - Print Working Directory \\
        Prints the directory you're currently on.
    \item \texttt{ls} - List \\
        List the items contained in a given directory.
    \item \texttt{cd} - Change Directory \\
        Move to another working directory.
    \item \texttt{mkdir} - Make Directory \\
        Create a new directory with the specified name.
    \item \texttt{touch} - Create File \\
        Create an empty file or update the timestamp of an existing file.
    \item \texttt{mv} - Move/Rename \\
        Move a file or directory to a new location or rename it.
    \item \texttt{cp} - Copy \\
        Copy files or directories to a specified location.
    \item \texttt{rm} - Remove \\
        Delete files or directories (use \texttt{-r} for recursive deletion).
    \item \texttt{echo} - Print Text \\
        Print text or the value of a variable to the terminal.
    \item \texttt{cat} - Concatenate and Display \\
        Display the content of a file or combine multiple files.
\end{enumerate}

\pagebreak
We can use these commands as follows (note that code starting with
\texttt{\$} is a common convention to indicate Bash commands):

\begin{verbatim}
# Start off by printing our current working directory
$ pwd
/Users/MyUser

# List the files and directories contained in /MyUser
$ ls
Applications    Downloads    Music       bin
Desktop         Library      Pictures    miniconda3
Documents       Public

# We can pass flags to most commands to access additional options
$ ls -l  # List items with detailed information
drwxr-xr-x   3 MyUser  staff    96 Jan 13 09:09 Applications
drwx------@  8 MyUser  staff   256 Jan 21 19:43 Desktop
drwx------+ 12 MyUser  staff   384 Jan 22 08:45 Documents
drwx------+  9 MyUser  staff   288 Jan 21 17:03 Downloads
drwx------@ 98 MyUser  staff  3136 Aug  5 14:33 Library
drwx------+  4 MyUser  staff   128 Dec 17  2023 Music
drwx------+  5 MyUser  staff   160 Dec 24 12:57 Pictures
drwxr-xr-x+  4 MyUser  staff   128 Nov 30  2023 Public
drwxr-xr-x@  2 MyUser  staff    64 Oct 14 08:16 bin
drwxr-xr-x  17 MyUser  staff   544 Nov 30  2023 miniconda3

# Let's move into /Documents using a relative path
$ cd ./Documents  # "." represents the current directory

# We could have achieved the same thing using an absolute path
$ cd /Users/MyUser/Documents/

# We can create a new directory
$ mkdir my-project

# We can create a new file inside the new folder
$ touch my-project/file.txt

# Let's move file.txt back to Documents and rename it
$ mv my-project/file.txt renamed-file.txt

# Let's copy renamed-file.txt back into my-project and rename it
$ cp renamed-file.txt my-project/copied-file.txt
\end{verbatim}

\pagebreak
\begin{verbatim}
# Let's delete renamed-file.txt
$ rm renamed-file.txt

# We can pass some text to a file with pipes
$ echo "Hello world\!" > my-project/copied-file.txt

# Let's display the file's contents to check if it worked
$ cat my-project/copied-file.txt
Hello world!

# We can append new text to the file >>
$ echo "A new line of text" >> my-project/copied-file.txt

# Let's see how the file changed
$ cat my-project/copied-file.txt
Hello world!
A new line of text

# We can replace the file's contents using a single pipe
$ echo "Starting over\!" > my-project/copied-file.txt

# Instead of appending a new line, we overwrote the whole file
$ cat my-project/copied-file.txt
Starting over!
\end{verbatim}

% Git
\section{Git}
Git is a Version Control System (VCS) that is used all around the world to
develop software. It allows developers to track changes to their code,
collaborate and manage different versions of a project. Whether you're working
alone as a researcher or in a large engineering team, Git helps maintain a
clear history of modifications and experiment with new features without risking
the integrity of preceding versions of your code.

At its core, Git works like checkpoints in video games. That is, you can save
your progress and go back any time you want. Additionally, you can set multiple
checkpoints (as many as you want) so you can go back to revert your code to any
previous version (not just the latest checkpoint). Moreover, Git allows you to
explore \textit{alternative timelines} known as \textit{branches} where you can
develop new features without affecting the functionality of previous versions.

Here are some of the main commands we'll be using.

\begin{enumerate}

    \item \texttt{git init} \\
        Initializes a new Git repository in the current directory (hidden folder
        named \texttt{.git}).

    \item \texttt{git clone <repository-url>} \\
        Clones an existing repository from a remote source (i.e., GitHub) to
        your local machine.

    \item \texttt{git add <file>} \\
        Stages changes to the specified file(s) for the next commit.

    \item \texttt{git commit -m "<message>"} \\
        Commits the staged changes with a descriptive message.

    \item \texttt{git status} \\
        Shows the current status of the working directory, including staged,
        unstaged, and untracked files.

    \item \texttt{git log} \\
        Displays the commit history of the repository. Use
        \textttt{git log --oneline} for improved readability.

    \item \texttt{git branch} \\
        Lists all branches in the repository. Use git branch <branch-name>` to
        create a new branch.

    \item \texttt{git checkout <branch-name>} \\
        Switches to the specified branch. Use
        \texttt{git checkout -b <branch-name>} to create and switch to a new
        branch using a single command.

    \item \texttt{git merge <branch-name>} \\
        Merges the specified branch into the current branch.

    \item \texttt{git pull} \\
        Fetches and integrates changes from the remote repository into the
        current branch.

    \item \texttt{git push} \\
        Uploads your local commits to the remote repository.

    \item \texttt{git remote -v} \\
        Lists the remote repositories linked to the local repository.

    \item \texttt{git stash} \\
        Temporarily saves changes in your working directory that have not yet
        been committed. Useful for switching to another branch without
        committing your changes.

    \item \texttt{git stash apply} \\
        Reapplies the most recent stashed changes.

    \item \texttt{git reset <file>} \\
        Unstages a file (in other words, it undoes \texttt{git add <file>}).

    \item \texttt{git revert <commit-hash>} \\
        Creates a new commit that undoes the changes from the specified commit.

    \item \texttt{git diff} \\
        Shows the differences between the working directory and the staging
        area.

    \item \texttt{git rm <file>} \\
        Removes a file from the working directory and stages its deletion (the
        removal must be committed afterwards!).

    \item \texttt{git tag <tag-name>} \\
        Creates a tag to mark a specific point in the repository’s history.

    \item \texttt{git fetch} \\
        Downloads changes from the remote repository without integrating them
        into the current branch.

\end{enumerate}

% End document 
\end{document}
